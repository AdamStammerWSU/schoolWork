% The "%" character denotes a comment
\documentclass[prb,preprint]{revtex4-1}
\usepackage{amsmath}  % needed for \tfrac, \bmatrix, etc.
\usepackage{amsfonts} % needed for bold Greek, Fraktur, and blackboard bold
\usepackage{graphicx} % needed for figures

\begin{document}
\title{Computer Conduction}
\author{Adam Stammer}
%\email{adam.stammer@go.winona.edu}

\date{\today}
\maketitle
%title page ends here



It may seem cliché, but computers are everywhere. From school and businesses to the home, life is becoming increasingly more automated. Even in my life, I've felt the growing presence of computers and experienced first had how powerful they can, and also how dangerous. In the immortal words of Ben Parker, "With great power comes great responsibility." 

For as long as I can remember I've always had an extreme desire to figure out how things work. From a common pen, to the TV remote control, big, small, it didn't matter, I wasn't satisfied until I knew what made it tick. On my 6th birthday my father came home with a box for me. Inside I found a Gameboy Color with a handful of games. I couldn't've been happier! I played that thing almost everyday for years, but it wasn't just the games that had me so intrigued. How did this magic box, turn electricity into entertainment? At the time, I didn't have the skillset to answer that question but I took that Gameboy apart and put it back together more times than I can count anyway, hoping it would eventually make sense. 

In middle school, a good friend of mine showed me a windows batch program he had written. It was my first experience with programming and I was hooked. Together we pushed each other to learn more languages, make more complex programs, and eat up all there was to know about computer software. We even entered some computer programming contests at local colleges. The more we learned, the more we realized how little we actually knew, and that's when I realized it was the perfect career area for me. What could better than getting paid to learn?



%\begin{thebibliography}{99}
% The numeral (here 99) in curly braces is nominally the number of entries in
% the bibliography. It's supposed to affect the amount of space around the
% numerical labels, so only the number of digits should matter--and even that
% seems to make no discernible difference.
%Not Requested
%\end{thebibliography}

\end{document}
