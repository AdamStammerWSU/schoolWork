% The "%" character denotes a comment
\documentclass{article}
\usepackage{amsmath}  % needed for \tfrac, \bmatrix, etc.
\usepackage{amsfonts} % needed for bold Greek, Fraktur, and blackboard bold
\usepackage{graphicx} % needed for figures

\begin{document}
\title{Computing Carbon}
\author{Adam Stammer}
%\email{adam.stammer@go.winona.edu}
\date{}
\maketitle
%title page ends here



Computers have never been as integrated into our daily lives as they are now. From our work to our home, life is becoming increasingly more automated. Even in my life, I've felt the growing presence of computers, and experienced first hand how powerful they can be. That power, however, comes with a responsibility to use computers the right way. I hope to use my knowledge and experience to create and implement more carbon efficient computers, while maintaining a pinnacle of ethical computing practices.

For as long as I can remember I've always had an extreme desire to figure out how things work. From a common pen, to the TV remote control, big, small, it didn't matter; I wasn't satisfied until I knew what made it tick. On my 6th birthday my father came home with a box for me. Inside I found a Gameboy Color with a handful of games. I couldn't've been happier! I played that thing almost everyday for years, but it wasn't just the games that had me so intrigued. How did this magic box turn electricity into entertainment? At the time, I didn't have the skillset to answer that question but I took that Gameboy apart and put it back together more times than I can count anyway, hoping it would eventually make sense. Little did I know, it planted the seeds for what I would one day pursue.

In middle school, a good friend of mine showed me a windows batch program he had written. It was my first experience with programming, and I was hooked. Together we pushed each other to learn more languages, make more complex programs, and eat up all there was to know about computer software. We even entered some computer programming contests at local colleges. The more we learned, the more we realized how little we actually knew, and that's when I realized it was the perfect career choice for me. What could be better than getting paid to learn? That brings the fun, but where does one find a legacy they can be proud of? I want to leave this world not just believing that I did more good than bad, but knowing that I had a lasting positive impact on it.

Entering college I had to seriously ask myself, precisely, what I wanted to accomplish with my skills and abilities. Thankfully, this question is an easy one for me. Computers are powerful tools, and like any such tool, if used unethically, they can cause significant damage. An axe used to cut the firewood that warms a persons home, can just as easily be used as a weapon. First and foremost, I want to do everything I can to make sure computers are used ethically. Such a thing is much easier said than done, but that just makes the task all the more vital. It remains a central pillar in my goals, but I still needed to get specific.

After reading an article that came out of the MIT Energy Initiative\footnote{http://energy.mit.edu/news/energy-efficient-computing/}, I quickly grew intrigued by Green Computing, a relatively new area of study aimed at making computers more efficient. The resources of our world are limited, and it's important that we use them sustainably. By some standards it's a luxury, by others it's a necessity. Imagine how much coal and oil we could save if every computer in the world used half as much electricity as they do right now. Think of the global effects if computers could be built with just a fraction of the materials, or if more parts of a computer were practical to recycle, and lasted longer in the first place. Something as simple as more efficient code has been seen to have significant impacts on the environmental costs of operating a computer. With computing growing to be more and more a part of our everyday lives, the impacts are certainly nontrivial. 

On the flip side of the coin, sometimes spending a little now can go a long way in the future. Computers can be made more efficient, but they can also be used to make other things more efficient. Smart Grids, for example, use computation and automation to decrease the overhead costs of maintaining an electrical grid. Something as simple as a programmable thermostat has significant impacts on one's carbon footprint. Computers are more integrated into daily life than ever before, but there's still so much more we can achieve with them!

To achieve these goals I plan on continuing my education in graduate school by pursuing a Doctorate in Computer Engineering, with research interests focused heavily on Green Computing. From there I hope to find new and better ways to decrease the environmental costs of computing and apply computers in efficient ways, while also inspiring and teaching the next generation to do even better. Computers are some of the most powerful tools at our disposal, so let's use them to build a better tomorrow.


%\begin{thebibliography}{99}
% The numeral (here 99) in curly braces is nominally the number of entries in
% the bibliography. It's supposed to affect the amount of space around the
% numerical labels, so only the number of digits should matter--and even that
% seems to make no discernible difference.
%Not Requested
%\end{thebibliography}

\end{document}
