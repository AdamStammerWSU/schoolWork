\documentstyle[12pt]{article}
\setlength{\oddsidemargin}{0in}
\setlength{\evensidemargin}{0in}
\setlength{\textwidth}{6.5in}
\setlength{\topmargin}{-.3in}
\setlength{\textheight}{9in}
\pagestyle{empty}



\newcommand{\university}{University of Minnesota }
\newcommand{\uni}{UMN }



\begin{document}

\begin{center}
{\Large \huge \scshape Statement of Purpose (DRAFT \#1)} \\[.3in]
{\large Adam R. Stammer}
\end{center}

\vspace*{.1in}

My primary interests in computer engineering are architecture, embedded systems,
and green computing. I first started many of these, and related, topics in early middle school
and have only grown since. My undergraduate study at Winona State University has allowed me
to grow that knowledge and experience, and apply it to projects in and out of class.

At a young age I found a strong passion for learning and figuring out how things worked. My
early exposure to electronics and computers quickly grew to overlap with the majority of my
hobbies because of the vast possible applications and sub topics. These interests were
fostered by various programming competitions I participated in during middle and high school including 
Dakota State University's ACM Programming Competition as well as South Dakota State University's Program
and Design Competition, the latter of which my team placed within the top three each of the four years I
participated.

In high school I became increasingly aware and concerned with Environmentalism and Sustainability.
Interested in how these topics might be combined with my other passions I was exposed to the field
of Green Computing. The long term impact I have on this world is important to me, and I want to leave
this world better than when I entered it. There are many aspects to this goal, but I plan on applying the 
skills and knowledge that I have and will continue to accumulate. 

Early on in my undergraduate studies at WSU I found a place within the Physics department, including 
leadership positions in Physics Club and various study groups. This networking and exposure led me to
the research project that I've been working on for the last 18 months. Under the direction of Dr. Carl
Ferkinhoff, I've been designing and building the electrical foundation for the Hardware.astronomy 
Housekeeping Box (H.aHkBox), as to be used with the ZEUS2 grating spectrometer. This project has allowed
me to hone and apply my skills in electrical engineering, embedded systems, and operating system design
with green computing considerations prevalent throughout. I also had the opportunity to formally present
this project at the 235th American Astronomical Society meeting in Honolulu, HI early this year.

My experience tutoring and leading study groups at WSU has also exposed me to the joy of teaching. Learning
is one of my strongest passions and being able to share that passion with other people is one of the 
most satisfying experiences I know. I've found interacting with other people to be extremely valuable in so many ways.
They expose us to new experiences and allow us to see things from new perspectives. Passing knowledge on,
and accepting that which others offer, can have a profound impact on how we all live and feel, as well as on the legacy
we leave behind. 

I plan to continue my study of architecture and embedded systems in graduate school, with a focus on
environmentally efficient interactions between digital hardware and software. I believe that the \university
is an ideal place to study these fields because \uni offers an excellent selection of relevant courses, 
and the community is filled with people that will help me foster and apply my interests. I love to teach and 
learn from others, and hope to apply and grow my abilities through research and learning at \uni.

\end{document}
