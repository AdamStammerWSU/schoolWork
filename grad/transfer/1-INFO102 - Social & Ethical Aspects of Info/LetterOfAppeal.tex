% The "%" character denotes a comment
\documentclass{article}
\usepackage{amsmath}  % needed for \tfrac, \bmatrix, etc.
\usepackage{amsfonts} % needed for bold Greek, Fraktur, and blackboard bold
\usepackage{graphicx} % needed for figures

\begin{document}


\begin{center}
	INFO102 Letter of Appeal
\end{center}

At SDSU, INFO102 was considered a recommended class for STEM Majors to meet their equivalent of a general ethics goal. Each week we spent one day in lecture learning about various ethical structures like utilitarianism, virtue theory, etc. as well as contextual ethics like personal, common, and professional ethics. Our other meeting day for each week we'd be posed an ethical problem, split up into groups, and discuss the situation. These situations were generally realistic STEM centered examples and accompanied a question like "Who's fault is it?", "What is the right thing to do here?", etc. As the course name implies, the ethical questions often involved privacy and digital information.

One situation I remember being quite polarizing: Company A designs and manufactures a robotic arm to assist on assembly lines. Company B purchases this robotic arm and installs it to be used on their line. The robotic arm malfunctions and injures a line worker. Jim, Tammy, and Frank designed the robotic arm. Fred and Terry manufactured it. John is the boss of all of the people. They all work for Company A. Benjamin is the injured worker, and Bill is his boss. Bill is the one who decided to get the robotic arm, and he had Bill install it. Bill and Benjamin both work for Company B. Who is at fault for Benjamin's injury?

Clearly this situation is far simplified from reality, but it really drives home the point that ethics are complicated, messy, and frequently fluid. There is rarely an easy answer. The goal of this class was not to get people to agree with the ethical decisions of the professor, but to give students the tools they needed to ethically analyze a situation with as little bias as possible.
\newline
Class in Question: SDU:INFO102
\newline
Argued Goal: WSU:Goal 9

Arguments of Appeal:
\begin{enumerate}
	\item Course Goals Align
	\item This course met SDSU's equivalent general goal
	\item "Ethics" is literally in the course title
\end{enumerate}

%\begin{thebibliography}{99}
% The numeral (here 99) in curly braces is nominally the number of entries in
% the bibliography. It's supposed to affect the amount of space around the
% numerical labels, so only the number of digits should matter--and even that
% seems to make no discernible difference.
%Not Requested
%\end{thebibliography}

\end{document}
