\documentstyle[12pt]{article}
\setlength{\oddsidemargin}{0in}
\setlength{\evensidemargin}{0in}
\setlength{\textwidth}{6.5in}
\setlength{\topmargin}{-.3in}
\setlength{\textheight}{9in}
\pagestyle{empty}

\begin{document}

\begin{center}
{\Large Statement of Purpose} \\[.3in]
{\large Robert S. French}
\end{center}

\vspace*{.5in}

My primary interests in computer science are computer architecture and
compiler technology.  I have participated in several projects during
my undergraduate career at MIT that have provided me with experience
in these areas, and I have recently specialized in parallel processing
systems.

My interest in computer architecture was first kindled during the
design contest for the MIT course ``Computation Structures.''  The
contest consists of redesigning an 8-bit microcoded processor to
improve its performance.  Our two-person team constructed a 16-bit
design for the processor and achieved at speedup of over 750 times.
As part of the design, I implemented a microcode assembler, a
microcode simulator, various levels of logic simulators, and over 80
pages of microcode.  The contest taught me a great deal about
low-level system design and the tradeoffs involved in designing fast
systems.

My next architecture project was for the course ``Digital Systems
Laboratory,'' in which my partner and I implemented a hardware
simulator for a highly-parallel, fine-grained optical computer.  The
system was based on recent research by Alan Huang at Bell Laboratories
in ``computational origami,'' a simple method of simulating an
arbitrarily large number of processors using a small set of processing
elements with long delay lines.  As part of the project I designed a
compiler and a placement and routing system for allocating processors.
I presented a paper on the algorithm at the MIT-ACM Undergraduate
Computer Science Conference in April, 1989, and was invited to present
my findings at Bell Labs.

I designed and implemented part of {\em Zephyr\/}, a distributed
multicast message system, as an employee of MIT's Project Athena.
Project Athena is an eight year experiment in the use of computers in
education, and consists of approximately 1,000 workstations connected
by a campus-wide network.  This setup provides a unique environment in
which to develop scalable distributed software.  I investigated
various communications methods, and was the primary architect of the
protocol and the authentication system used by {\em Zephyr\/}.  Our
development team published a paper on {\em Zephyr\/} in the Winter
1988 Usenix Conference Proceedings.

My current research is for my undergraduate thesis, in which I am
investigating the use of fractals to measure communication locality in
distributed systems.  I hope to extend this research and develop new
algorithms for efficient processor allocation.

I plan to continue my study of architecture and compiler technology in
graduate school, and I feel that Stanford is the ideal place to study
these fields.  Stanford offers an excellent selection of courses and
supports a vigorous research program in my areas of interest.  I enjoy
developing theories and building hardware systems on which to test
them, and hope to be involved in applied research at Stanford.

After receiving a Ph.D., I plan to pursue an academic career in
computer science research.

\end{document}
