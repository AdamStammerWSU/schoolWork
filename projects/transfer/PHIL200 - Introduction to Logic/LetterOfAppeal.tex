% The "%" character denotes a comment
\documentclass{article}
\usepackage{amsmath}  % needed for \tfrac, \bmatrix, etc.
\usepackage{amsfonts} % needed for bold Greek, Fraktur, and blackboard bold
\usepackage{graphicx} % needed for figures

\begin{document}


\begin{center}
	PHIL200 Letter of Appeal
\end{center}

At SDSU, PHIL200 is considered a good alternative to Introduction to Philosophy for fulfilling the equivalent general goal. I thought it would pair well with my CS degree, and I wasn't disappointed. We looked at formal and informal logic. Throughout the semester we worked through a book with lectures and numerous examples. The professor emphasized that logic is a powerful tool to finding the truth, but it is not truth itself. A logical statement is not inherently true. For example, "A lion has 4 legs. My cat has 4 legs. Therefore, my cat is a lion." This statement is logical, but I'm quite certain it's not true. Understanding this really helped me get a better understanding of the world and why it can be so difficult for a group of people to agree on something, even when that something seems concrete, objective, and obvious. We also went through numerous fallacies of logic so that we might avoid them, and that they would aid in effectively dissuading others from a logically faulty argument.
\newline
Class in Question: SDU:PHIL200
\newline
Argued Goal: WSU:Goal 6

Arguments of Appeal:
\begin{enumerate}
	\item Course Goals Align
	\item This course satisfies SDSU's equivalent general goal
\end{enumerate}

%\begin{thebibliography}{99}
% The numeral (here 99) in curly braces is nominally the number of entries in
% the bibliography. It's supposed to affect the amount of space around the
% numerical labels, so only the number of digits should matter--and even that
% seems to make no discernible difference.
%Not Requested
%\end{thebibliography}

\end{document}
